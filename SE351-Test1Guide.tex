\documentclass[10pt,a4paper]{article}
\usepackage[latin1]{inputenc}
\usepackage{amsmath}
\usepackage{hyperref}
\usepackage{amsfonts}
\usepackage{amssymb}
\usepackage{graphicx}
\usepackage{fancyhdr}
\usepackage{titling}
\usepackage{lastpage}
\usepackage[margin=1in]{geometry}
\usepackage{times}

\author{Michael Kemp}
\title{SOFTENG 351: Fundamentals of Database Systems - Guide to Test 1}
\date{\today}

\lhead{\thetitle}
\rhead{Section: \thesection}
\lfoot{Michael Kemp}
\cfoot{Version: ALPHA\#1 as of \thedate}
\rfoot{Page \thepage \ of \pageref{LastPage}}
\pagestyle{fancy}
\renewcommand{\footrulewidth}{0.4pt}

\begin{document}
\maketitle
\tableofcontents
\newpage


\section{Basics of the relational model of data}
	\subsection{The relational model of data}
		\begin{itemize}
			\item \textit{relations} are sets of \textit{tuples} often represented as a table
			\item \textit{attributes} are the column titles of a relation
			\item for each attribute we assign a \textit{domain} which is a universal set containing all possible values(like a string; $dom(A) = string$
			\item \textit{tuples} are the rows of a relation and all have the same structure in a relation
			\item if there is no value for an attribute then the value is \textit{null}
			\item \textit{relation schema} are a finite set $R$ where attributes are $A$ and each attribute $A\in R$ has a domain $dom(A)$
			\item relation schema can be written $R = \{A_1, A_2 ... A_n\}$ or $R(A_1, A_2 ... A_n)$ or $R(A_1:dom(A_1) ... A_n:dom(A_n)$
			\item All \textit{R-tuples} (a tuple in a relation schema) are an element $t$ of the Cartesian product of the domains of all the attributes $t \in A_1 \times A_2 \times ... A_n$ because each attribute's value is bound to it's respective domain.
			\item \textit{$R$-relations} are a finite set $r$ of $R$-tuples thus $r\subseteq dom(A_1)\times...dom(A_n)$
			\item R-tuples can be written with their values $t = (A_1:v_1...A_n:v_n)$
			\item A \textit{database-schema} is a finite set $S$ of relation schemata
			\item An $S$-database $I$ consists of one $R$-relation for $I(R)$ for each relation $R$ in $S$ ($I = \{I(R)|R\in S\}$)
			\item Having duplicates in a database is normally useless so we have \textit{keys} to ensure a uniqueness over an attribute or a combination of
			\item \textit{superkey} over a relation schema $R$
			\begin{itemize}
				\item finite, non-empty subset $K \subseteq R$
				\item is \textit{satisfied} if an $R$-relation $r$ only has tuples with a unique combination of values for each attribute in the superkey.
				\item A \textit{Key} is a superkey if there is no subset which is also a satisfied superkey
				\item A \textit{foreign key} is when all of the combination of values of attributes(in the foreign key) is in the set of the table which defines the foreign key($[A_1..A_n] \subseteq S[A_1..A_n]$). Also the same $S$ can not be referenced twice.
			\end{itemize}
		\end{itemize}
	
	\subsection{SQL as a data definition and query language}
	this whole section is not in the test
		\begin{itemize}
			\item is a DDL (Data Definition Language) and a DML (Data Manipulation Language)
			\item names are not case sensitive
			\item DDL
			\begin{itemize}
				\item CREATE TABLE $<table name> <attribute1 domain, ...>$ attributes can be specified as NOT NULL
				\item DROP TABLE $<table name>$
				\item ALTER TABLE followed by other stuff
			\end{itemize}
			\item Domains
			\begin{itemize}
				\item CHARACTER: Character strings of set length
				\item VARCHAR: Character strings up to set length
				\item NATIONAL CHARACTER
				\item INTEGER: Signed 32-bit integer
				\item SMALLINT: Signed 16-bit integer
				\item NUMERIC = DECIMAL: Numeric values with definable precision and scale
				\item FLOAT: Approximate numeric values with definable precision(up to 64)
				\item REAL: Approximate numeric values up to 64 precision
				\item DOUBLE PRECISION: Double a REAL
				\item BIT = BOOLEAN: TRUE, FALSE, false or true
				\item BIT VARYING: Bit strings
				\item DATE: YYYY-MM-DD but can use single digits for month and date
				\item TIME: HH:MM:SS with optional nano seconds and seconds up to (including) 61.999999
			\end{itemize}
			\item to insert a tuple to a relation all values must be known thus null exists if it doesn't exist or is no yet known
			\item \textit{duplicate tuples} are tuples where both values are the same for every attribute
			\item duplicate tuples are not allowed in relations but IRL it's allowed because it's computational expensive to manage.
			\item A table is $X$-total if all it's tuples are $X$-total
			\item Constraints
			\begin{itemize}
				\item NOT NULL means $A$ must be $A$-total
				\item 
			\end{itemize}			
		\end{itemize}
		I'll come back to this

\section{Query Languages}
	\subsection{Relational algebra}
		\begin{itemize}
			\item $A$ is the set of possible relations
			\item Partial operations on $A$ take either 1 or 2 relations as input and produce another relation as output
			\item $\#r$ is the set of attributes of a relation $r$
			\item Operations:
			\begin{itemize}
				\item Attribute selection: $\sigma_{A=B}$ produces a relation where tuples have the same value in attribute $A$ as in $B$
				\item Constant selection: $\sigma_{A=c}$ Same as attribute but against a constant $c$ instead of an attribute
				\item Projection: $\pi_{A_1,...A_n}(r)$ Takes a relation $r$ and returns another with only the specified attributes $A_1,...A_n$
				\item Renaming: $\delta_{oldname\mapsto newname}$ Changes the name of an attribute without changing the relation
				\item Union: $r_1\cup r_2$ Relations with the same set of attributes $\#r_1 = \#r_2$ form a relation with tuples from both
				\item Difference: $r_1 - r_2$ Relations with the same set of attributes $\#r_1 = \#r_2$ form a relation with the tuples from the first relation $r_1$ that aren't in the second relation $r_2$
				\item Join (Natural): $r_1 \bowtie r_2$ Joins tuples in both relations where attributes that are in both all have the same values. If there is more than one match then the cross product is given
			\end{itemize}
			\item Redundant Operations:
			\begin{itemize}
				\item Intersection: $r_1\cap r_2$ Returns a relation where tuples are in both relations $r_1, r_2$ and have the same set of attributes $\#r_1 = \#r_2$
				\item Cross-product: $r_1\times r_2$ where both relation have no common attributes the produced relation is every possible pair of tuples
				\item Division: $r_1\div r_2$ the set of attributes in $r_2$ must be a superset of $r_1$. Where there is there is attribute values for every tuple in $r_2$. It's hard to explain and much easier to show. \url{http://www.mathcs.emory.edu/~cheung/Courses/377/Syllabus/4-RelAlg/division.html}
			\end{itemize}
		\end{itemize}

	\subsection{Relational calculus}
		\begin{itemize}
			\item Tuple Relational Calculus is where variables represent tuples
			\item Domain Relation Calculus (DRC) is where variables represent individual values in the domains $D_i$
			\item We focus on DRC
			\item Objects(terms): are placeholders(variables) and values(constants)
			\item I will come back to this
			\item $\mathcal{F}$ is the set of all formulae
			\item Shortcuts:
			\begin{itemize}
				\item Inequation: $t\neq t' \Leftrightarrow \neg(t=t')$
				\item Disjunction: $\psi\vee\phi \Leftrightarrow \neg(\neg\psi\wedge\neg\phi)$
				\item Universal quantification: $\forall x(\phi) \Leftrightarrow \neg\exists x(\neg\phi)$
				\item Implication: $\phi\Rightarrow\psi \Leftrightarrow \neg\phi\vee\psi$
				\item Equivalence: $(\psi\Leftrightarrow\phi) \Leftrightarrow  (\phi\Rightarrow\psi)\wedge(\psi\Rightarrow\phi)$
				\item Successive quantification: $\forall x_1(\forall x_2(...(\psi))) \Leftrightarrow \forall x_1,x_2,...(\psi)$ same with $\exists$
			\end{itemize}
			\item Free variables are those not bound by a quantifier 
			\item I'll come back to this....right?.....right?
			\item Domain Relational Calculus
			\begin{itemize}
				\item Every query $Q$ in the language of DRC $\mathcal{L}_{DRC}$ is expressed in the form $Q = \{(x_1,...,x_n)|\varphi\}$
				\item query mapping on a database is expressed as $\{(list of variables)|Q\}$
			\end{itemize}
			\item Safe Range Normal Form (SRNF)
			\begin{itemize}
				\item Doesn't change the query; just makes it safe
				\item Elimination of shortcuts: remove all universal quantification, implication and equivalence
				\item Bounded renaming: change names of bound variables so there is non both bound and free
				\item Shift Negation: replace sub-formula until negation is only in front of quantification and atoms (down to atoms and remove double negation)
				\item Shift disjunction: no ands of ors; use de morgan's
				\item Omit parentheses: remove all unnecessary 
			\end{itemize}
		\end{itemize}
		tell the author to stop procrastinating...	
	
	\subsection{SQL}
	not in the test

\section{Database design}
	\subsection{Entity-Relationship modelling}
	not in the test

\end{document}